
\section{Conclusiones}

\begin{itemize}
    \item Problema 1. 
    \subitem El intervalo de confianza del 95\% nos indica que podemos estar 95\% seguros de que la verdadera media de ventas se encuentra dentro de ese rango. El intervalo de confianza del 99\% es más amplio pero nos da mayor certeza, indicando que podemos estar 99\% seguros de que la verdadera media está en ese rango.
    \subitem Los intervalos más amplios indican mayor variabilidad en las ventas de ese mes. Los intervalos más estrechos indican ventas más consistentes. 
    \subitem Febrero mostró la mayor variabilidad en las ventas, lo que podría indicar mayor incertidumbre en las predicciones para este mes.
    \subitem Esta información puede ser útil para la planificación de inventario y recursos, prestando especial atención a los meses con mayor variabilidad.
    \item Problema 2. 
    \subitem Dado que el valor $p$ ($2.3354 \times 10^{-58}$) es considerablemente menor que el nivel de significancia $\alpha = 0.05\%$, se rechaza la hip\'otesis nula ($H_0$). Esto implica que existe al menos una tienda cuyas ventas promedio son estad\'isticamente diferentes de las dem\'as.
    \subitem Tambi\'en se puede observar con el diagrama de cajas como las ventas de cada tienda se distribuyen de forma similar pero no identicas, asi como lo confirma el analisis ANOVA
    \item Problema 3. 
    \subitem La diferencia es estad\'isticamente significativa ($p = 8.6900 \times 10^{-51}$). Por lo tanto, se rechaza la hip\'otesis nula ($H_0$) y se concluye que hay una diferencia significativa entre las medias de las ventas de las tiendas C\'ordoba y La Floresta.
\end{itemize}

